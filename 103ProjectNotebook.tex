% Options for packages loaded elsewhere
\PassOptionsToPackage{unicode}{hyperref}
\PassOptionsToPackage{hyphens}{url}
%
\documentclass[
]{article}
\usepackage{amsmath,amssymb}
\usepackage{lmodern}
\usepackage{iftex}
\ifPDFTeX
  \usepackage[T1]{fontenc}
  \usepackage[utf8]{inputenc}
  \usepackage{textcomp} % provide euro and other symbols
\else % if luatex or xetex
  \usepackage{unicode-math}
  \defaultfontfeatures{Scale=MatchLowercase}
  \defaultfontfeatures[\rmfamily]{Ligatures=TeX,Scale=1}
\fi
% Use upquote if available, for straight quotes in verbatim environments
\IfFileExists{upquote.sty}{\usepackage{upquote}}{}
\IfFileExists{microtype.sty}{% use microtype if available
  \usepackage[]{microtype}
  \UseMicrotypeSet[protrusion]{basicmath} % disable protrusion for tt fonts
}{}
\makeatletter
\@ifundefined{KOMAClassName}{% if non-KOMA class
  \IfFileExists{parskip.sty}{%
    \usepackage{parskip}
  }{% else
    \setlength{\parindent}{0pt}
    \setlength{\parskip}{6pt plus 2pt minus 1pt}}
}{% if KOMA class
  \KOMAoptions{parskip=half}}
\makeatother
\usepackage{xcolor}
\usepackage[margin=1in]{geometry}
\usepackage{color}
\usepackage{fancyvrb}
\newcommand{\VerbBar}{|}
\newcommand{\VERB}{\Verb[commandchars=\\\{\}]}
\DefineVerbatimEnvironment{Highlighting}{Verbatim}{commandchars=\\\{\}}
% Add ',fontsize=\small' for more characters per line
\usepackage{framed}
\definecolor{shadecolor}{RGB}{248,248,248}
\newenvironment{Shaded}{\begin{snugshade}}{\end{snugshade}}
\newcommand{\AlertTok}[1]{\textcolor[rgb]{0.94,0.16,0.16}{#1}}
\newcommand{\AnnotationTok}[1]{\textcolor[rgb]{0.56,0.35,0.01}{\textbf{\textit{#1}}}}
\newcommand{\AttributeTok}[1]{\textcolor[rgb]{0.77,0.63,0.00}{#1}}
\newcommand{\BaseNTok}[1]{\textcolor[rgb]{0.00,0.00,0.81}{#1}}
\newcommand{\BuiltInTok}[1]{#1}
\newcommand{\CharTok}[1]{\textcolor[rgb]{0.31,0.60,0.02}{#1}}
\newcommand{\CommentTok}[1]{\textcolor[rgb]{0.56,0.35,0.01}{\textit{#1}}}
\newcommand{\CommentVarTok}[1]{\textcolor[rgb]{0.56,0.35,0.01}{\textbf{\textit{#1}}}}
\newcommand{\ConstantTok}[1]{\textcolor[rgb]{0.00,0.00,0.00}{#1}}
\newcommand{\ControlFlowTok}[1]{\textcolor[rgb]{0.13,0.29,0.53}{\textbf{#1}}}
\newcommand{\DataTypeTok}[1]{\textcolor[rgb]{0.13,0.29,0.53}{#1}}
\newcommand{\DecValTok}[1]{\textcolor[rgb]{0.00,0.00,0.81}{#1}}
\newcommand{\DocumentationTok}[1]{\textcolor[rgb]{0.56,0.35,0.01}{\textbf{\textit{#1}}}}
\newcommand{\ErrorTok}[1]{\textcolor[rgb]{0.64,0.00,0.00}{\textbf{#1}}}
\newcommand{\ExtensionTok}[1]{#1}
\newcommand{\FloatTok}[1]{\textcolor[rgb]{0.00,0.00,0.81}{#1}}
\newcommand{\FunctionTok}[1]{\textcolor[rgb]{0.00,0.00,0.00}{#1}}
\newcommand{\ImportTok}[1]{#1}
\newcommand{\InformationTok}[1]{\textcolor[rgb]{0.56,0.35,0.01}{\textbf{\textit{#1}}}}
\newcommand{\KeywordTok}[1]{\textcolor[rgb]{0.13,0.29,0.53}{\textbf{#1}}}
\newcommand{\NormalTok}[1]{#1}
\newcommand{\OperatorTok}[1]{\textcolor[rgb]{0.81,0.36,0.00}{\textbf{#1}}}
\newcommand{\OtherTok}[1]{\textcolor[rgb]{0.56,0.35,0.01}{#1}}
\newcommand{\PreprocessorTok}[1]{\textcolor[rgb]{0.56,0.35,0.01}{\textit{#1}}}
\newcommand{\RegionMarkerTok}[1]{#1}
\newcommand{\SpecialCharTok}[1]{\textcolor[rgb]{0.00,0.00,0.00}{#1}}
\newcommand{\SpecialStringTok}[1]{\textcolor[rgb]{0.31,0.60,0.02}{#1}}
\newcommand{\StringTok}[1]{\textcolor[rgb]{0.31,0.60,0.02}{#1}}
\newcommand{\VariableTok}[1]{\textcolor[rgb]{0.00,0.00,0.00}{#1}}
\newcommand{\VerbatimStringTok}[1]{\textcolor[rgb]{0.31,0.60,0.02}{#1}}
\newcommand{\WarningTok}[1]{\textcolor[rgb]{0.56,0.35,0.01}{\textbf{\textit{#1}}}}
\usepackage{graphicx}
\makeatletter
\def\maxwidth{\ifdim\Gin@nat@width>\linewidth\linewidth\else\Gin@nat@width\fi}
\def\maxheight{\ifdim\Gin@nat@height>\textheight\textheight\else\Gin@nat@height\fi}
\makeatother
% Scale images if necessary, so that they will not overflow the page
% margins by default, and it is still possible to overwrite the defaults
% using explicit options in \includegraphics[width, height, ...]{}
\setkeys{Gin}{width=\maxwidth,height=\maxheight,keepaspectratio}
% Set default figure placement to htbp
\makeatletter
\def\fps@figure{htbp}
\makeatother
\setlength{\emergencystretch}{3em} % prevent overfull lines
\providecommand{\tightlist}{%
  \setlength{\itemsep}{0pt}\setlength{\parskip}{0pt}}
\setcounter{secnumdepth}{-\maxdimen} % remove section numbering
\ifLuaTeX
  \usepackage{selnolig}  % disable illegal ligatures
\fi
\IfFileExists{bookmark.sty}{\usepackage{bookmark}}{\usepackage{hyperref}}
\IfFileExists{xurl.sty}{\usepackage{xurl}}{} % add URL line breaks if available
\urlstyle{same} % disable monospaced font for URLs
\hypersetup{
  pdftitle={103ProjectNotebook},
  pdfauthor={Megan McKenzie},
  hidelinks,
  pdfcreator={LaTeX via pandoc}}

\title{103ProjectNotebook}
\author{Megan McKenzie}
\date{2023-08-08}

\begin{document}
\maketitle

\begin{enumerate}
\def\labelenumi{\arabic{enumi}.}
\tightlist
\item
  set up the data frames
\end{enumerate}

\begin{Shaded}
\begin{Highlighting}[]
\FunctionTok{setwd}\NormalTok{(}\StringTok{"../103RFinalProject"}\NormalTok{) }\CommentTok{\# set working directory}
\NormalTok{expressiondf }\OtherTok{\textless{}{-}} \FunctionTok{read.csv}\NormalTok{(}\AttributeTok{file =} \StringTok{"QBS103\_finalProject\_geneExpression.csv"}\NormalTok{, }\AttributeTok{header =} \ConstantTok{TRUE}\NormalTok{) }\CommentTok{\# read data from csv file to make df}
\NormalTok{metadf }\OtherTok{\textless{}{-}} \FunctionTok{read.csv}\NormalTok{(}\AttributeTok{file =} \StringTok{"QBS103\_finalProject\_metadata.csv"}\NormalTok{, }\AttributeTok{header =} \ConstantTok{TRUE}\NormalTok{) }\CommentTok{\# read data from csv file to make another df}
\end{Highlighting}
\end{Shaded}

convert expressiondf to long form to prep for linking

\begin{Shaded}
\begin{Highlighting}[]
\FunctionTok{library}\NormalTok{(tidyverse) }\CommentTok{\# I am using Tidyverse in this project}
\end{Highlighting}
\end{Shaded}

\begin{verbatim}
## -- Attaching core tidyverse packages ------------------------ tidyverse 2.0.0 --
## v dplyr     1.1.2     v readr     2.1.4
## v forcats   1.0.0     v stringr   1.5.0
## v ggplot2   3.4.1     v tibble    3.2.1
## v lubridate 1.9.2     v tidyr     1.3.0
## v purrr     1.0.1     
## -- Conflicts ------------------------------------------ tidyverse_conflicts() --
## x dplyr::filter() masks stats::filter()
## x dplyr::lag()    masks stats::lag()
## i Use the ]8;;http://conflicted.r-lib.org/conflicted package]8;; to force all conflicts to become errors
\end{verbatim}

\begin{Shaded}
\begin{Highlighting}[]
\NormalTok{longexpressiondf }\OtherTok{\textless{}{-}}\NormalTok{ expressiondf }\SpecialCharTok{\%\textgreater{}\%} \CommentTok{\# create a new df that is the long version of our original dataset}
  \FunctionTok{pivot\_longer}\NormalTok{(}\AttributeTok{cols =} \FunctionTok{starts\_with}\NormalTok{(}\FunctionTok{c}\NormalTok{(}\StringTok{"COVID"}\NormalTok{, }\StringTok{"NONCOVID"}\NormalTok{)),}
               \AttributeTok{names\_to =} \StringTok{"participant\_id"}\NormalTok{,}
               \AttributeTok{values\_to =} \StringTok{"gene\_expression\_data"}\NormalTok{)}
\end{Highlighting}
\end{Shaded}

\begin{enumerate}
\def\labelenumi{\arabic{enumi}.}
\setcounter{enumi}{1}
\tightlist
\item
  link data sets together to make 1 dataframe
\end{enumerate}

\begin{Shaded}
\begin{Highlighting}[]
\NormalTok{list\_df }\OtherTok{=} \FunctionTok{list}\NormalTok{(longexpressiondf, metadf) }\CommentTok{\# link the two dfs together by using list() then reduce()}
\NormalTok{df }\OtherTok{\textless{}{-}}\NormalTok{ list\_df }\SpecialCharTok{\%\textgreater{}\%} \FunctionTok{reduce}\NormalTok{(inner\_join, }\AttributeTok{by=}\StringTok{\textquotesingle{}participant\_id\textquotesingle{}}\NormalTok{)}
\end{Highlighting}
\end{Shaded}

\begin{enumerate}
\def\labelenumi{\arabic{enumi}.}
\setcounter{enumi}{2}
\tightlist
\item
  Generate the following 3 plots using ggplot2 for your covariates of
  choice:
\end{enumerate}

\begin{enumerate}
\def\labelenumi{(\alph{enumi})}
\tightlist
\item
  Histogram for gene expression
\end{enumerate}

\begin{Shaded}
\begin{Highlighting}[]
\FunctionTok{library}\NormalTok{(ggplot2) }\CommentTok{\# we are using ggplot2 in this project}

\NormalTok{subset }\OtherTok{\textless{}{-}} \FunctionTok{subset}\NormalTok{(df, X }\SpecialCharTok{==} \StringTok{\textquotesingle{}CLEC3B\textquotesingle{}}\NormalTok{) }\CommentTok{\# create a subset of the df that includes only the gene of interest}
\NormalTok{subset}
\end{Highlighting}
\end{Shaded}

\begin{verbatim}
## # A tibble: 125 x 27
##    X      participant_id             gene_expression_data geo_accession status  
##    <chr>  <chr>                                     <dbl> <chr>         <chr>   
##  1 CLEC3B COVID_01_39y_male_NonICU                   1.41 GSM4753021    Public ~
##  2 CLEC3B COVID_02_63y_male_NonICU                   0.79 GSM4753022    Public ~
##  3 CLEC3B COVID_03_33y_male_NonICU                   0.5  GSM4753023    Public ~
##  4 CLEC3B COVID_04_49y_male_NonICU                   0.7  GSM4753024    Public ~
##  5 CLEC3B COVID_05_49y_male_NonICU                   0.1  GSM4753025    Public ~
##  6 CLEC3B COVID_07_38y_female_NonICU                 0.33 GSM4753027    Public ~
##  7 CLEC3B COVID_08_78y_male_ICU                      0.61 GSM4753028    Public ~
##  8 CLEC3B COVID_09_64y_female_ICU                    0.77 GSM4753029    Public ~
##  9 CLEC3B COVID_10_62y_male_ICU                      1.69 GSM4753030    Public ~
## 10 CLEC3B COVID_11_52y_female_NonICU                 0.2  GSM4753031    Public ~
## # i 115 more rows
## # i 22 more variables: X.Sample_submission_date <chr>, last_update_date <chr>,
## #   type <chr>, channel_count <int>, source_name_ch1 <chr>, organism_ch1 <chr>,
## #   disease_status <chr>, age <chr>, sex <chr>, icu_status <chr>,
## #   apacheii <chr>, charlson_score <int>, mechanical_ventilation <chr>,
## #   ventilator.free_days <int>, hospital.free_days_post_45_day_followup <int>,
## #   ferritin.ng.ml. <chr>, crp.mg.l. <chr>, ddimer.mg.l_feu. <chr>, ...
\end{verbatim}

\begin{Shaded}
\begin{Highlighting}[]
\NormalTok{histogramplot }\OtherTok{\textless{}{-}} \FunctionTok{ggplot}\NormalTok{(subset, }\FunctionTok{aes}\NormalTok{(}\AttributeTok{x =}\NormalTok{ gene\_expression\_data)) }\SpecialCharTok{+} \CommentTok{\# build a histogram plot of gene expression and frequency}
  \FunctionTok{geom\_histogram}\NormalTok{(}\AttributeTok{color =} \StringTok{"pink"}\NormalTok{, }\AttributeTok{fill =} \StringTok{"magenta"}\NormalTok{) }\SpecialCharTok{+}
  \FunctionTok{labs}\NormalTok{(}\AttributeTok{title =} \FunctionTok{expression}\NormalTok{(}\FunctionTok{paste}\NormalTok{(}\FunctionTok{italic}\NormalTok{(}\StringTok{"CLEC3B"}\NormalTok{), }\StringTok{" Gene Expression"}\NormalTok{)), }\AttributeTok{y =} \StringTok{"Frequency"}\NormalTok{, }\AttributeTok{x =} \FunctionTok{expression}\NormalTok{(}\FunctionTok{paste}\NormalTok{(}\FunctionTok{italic}\NormalTok{(}\StringTok{"CLEC3B"}\NormalTok{), }\StringTok{" Gene Expression"}\NormalTok{))) }\SpecialCharTok{+}
    \FunctionTok{theme}\NormalTok{(}\AttributeTok{plot.title =} \FunctionTok{element\_text}\NormalTok{(}\AttributeTok{colour =} \StringTok{"orange1"}\NormalTok{, }\AttributeTok{face =} \StringTok{"bold"}\NormalTok{, }\AttributeTok{hjust =} \FloatTok{0.5}\NormalTok{)) }\CommentTok{\# for the plot title}

\NormalTok{histogramplot}
\end{Highlighting}
\end{Shaded}

\begin{verbatim}
## `stat_bin()` using `bins = 30`. Pick better value with `binwidth`.
\end{verbatim}

\includegraphics{103ProjectNotebook_files/figure-latex/unnamed-chunk-4-1.pdf}

\begin{enumerate}
\def\labelenumi{(\alph{enumi})}
\setcounter{enumi}{1}
\tightlist
\item
  Scatterplot for gene expression and continuous covariate
\end{enumerate}

\begin{Shaded}
\begin{Highlighting}[]
\NormalTok{scatterplot }\OtherTok{\textless{}{-}} \FunctionTok{ggplot}\NormalTok{(subset, }\FunctionTok{aes}\NormalTok{(}\AttributeTok{x =}\NormalTok{ gene\_expression\_data, }\AttributeTok{y =}\NormalTok{ ventilator.free\_days)) }\SpecialCharTok{+}  \CommentTok{\# build a scatter plot of gene expression and ventilator days}
  \FunctionTok{labs}\NormalTok{(}\AttributeTok{title =} \FunctionTok{expression}\NormalTok{(}\FunctionTok{paste}\NormalTok{(}\FunctionTok{italic}\NormalTok{(}\StringTok{"CLEC3B"}\NormalTok{), }\StringTok{"Gene Expression and Ventilator Days"}\NormalTok{)), }\AttributeTok{x =} \FunctionTok{expression}\NormalTok{(}\FunctionTok{paste}\NormalTok{(}\FunctionTok{italic}\NormalTok{(}\StringTok{"CLEC3B"}\NormalTok{), }\StringTok{" Gene Expression"}\NormalTok{), }\AttributeTok{y =} \StringTok{"Frequency"}\NormalTok{), }\AttributeTok{y =} \StringTok{"Ventilator Free Days"}\NormalTok{) }\SpecialCharTok{+}
  \FunctionTok{geom\_point}\NormalTok{(}\AttributeTok{color =} \StringTok{"purple"}\NormalTok{, }\AttributeTok{alpha =} \FloatTok{0.5}\NormalTok{) }\SpecialCharTok{+}
  \FunctionTok{theme}\NormalTok{(}\AttributeTok{plot.title =} \FunctionTok{element\_text}\NormalTok{(}\AttributeTok{colour =} \StringTok{"darkmagenta"}\NormalTok{, }\AttributeTok{face =} \StringTok{"bold"}\NormalTok{, }\AttributeTok{hjust =} \FloatTok{0.6}\NormalTok{)) }\CommentTok{\# for the plot title}
\NormalTok{scatterplot}
\end{Highlighting}
\end{Shaded}

\includegraphics{103ProjectNotebook_files/figure-latex/unnamed-chunk-5-1.pdf}

\begin{enumerate}
\def\labelenumi{(\alph{enumi})}
\setcounter{enumi}{2}
\tightlist
\item
  Boxplot of gene expression separated by both categorical covariates
\end{enumerate}

\begin{Shaded}
\begin{Highlighting}[]
\NormalTok{co }\OtherTok{\textless{}{-}} \FunctionTok{c}\NormalTok{(}\StringTok{"mistyrose2"}\NormalTok{, }\StringTok{"lavender"}\NormalTok{) }\CommentTok{\# make my own color palette}

\NormalTok{subset}\SpecialCharTok{$}\NormalTok{charlson\_binary }\OtherTok{\textless{}{-}}\NormalTok{ subset}\SpecialCharTok{$}\NormalTok{charlson\_score }\SpecialCharTok{\textgreater{}} \DecValTok{5} \CommentTok{\# make a new column that has T if the Charlson Score is extreme (more than 5) else it is F}

\NormalTok{boxplot }\OtherTok{\textless{}{-}} \FunctionTok{ggplot}\NormalTok{(subset, }\FunctionTok{aes}\NormalTok{(}\AttributeTok{x =}\NormalTok{ gene\_expression\_data, }\AttributeTok{y =}\NormalTok{ charlson\_binary, }\AttributeTok{color =}\NormalTok{ mechanical\_ventilation)) }\SpecialCharTok{+} \CommentTok{\#create a boxplot of gene expression, charlson score, and mechanical ventilation}
  \FunctionTok{geom\_boxplot}\NormalTok{() }\SpecialCharTok{+}
  \FunctionTok{geom\_jitter}\NormalTok{() }\SpecialCharTok{+}
  \FunctionTok{labs}\NormalTok{(}\AttributeTok{title =} \FunctionTok{expression}\NormalTok{(}\FunctionTok{paste}\NormalTok{(}\FunctionTok{italic}\NormalTok{(}\StringTok{"CLEC3B"}\NormalTok{), }\StringTok{"Gene Expression, Charlson Score, and Mechanical Ventilation"}\NormalTok{)), }\AttributeTok{x =} \FunctionTok{expression}\NormalTok{(}\FunctionTok{paste}\NormalTok{(}\FunctionTok{italic}\NormalTok{(}\StringTok{"CLEC3B"}\NormalTok{), }\StringTok{" Gene Expression"}\NormalTok{), }\AttributeTok{y =} \StringTok{"Frequency"}\NormalTok{), }\AttributeTok{y =} \StringTok{"Charlson Binary"}\NormalTok{) }\SpecialCharTok{+} 
  \FunctionTok{theme}\NormalTok{(}\AttributeTok{plot.title =} \FunctionTok{element\_text}\NormalTok{(}\AttributeTok{colour =} \StringTok{"orangered"}\NormalTok{, }\AttributeTok{face =} \StringTok{"bold"}\NormalTok{, }\AttributeTok{hjust =} \FloatTok{0.18}\NormalTok{, }\AttributeTok{vjust =} \SpecialCharTok{{-}}\DecValTok{2}\NormalTok{)) }\SpecialCharTok{+} \CommentTok{\# for the plot title}
  \FunctionTok{scale\_fill\_manual}\NormalTok{(}\AttributeTok{labels=}\FunctionTok{c}\NormalTok{(}\StringTok{"\textless{}5"}\NormalTok{, }\StringTok{"\textgreater{}5"}\NormalTok{)) }\SpecialCharTok{+}
  \FunctionTok{coord\_flip}\NormalTok{() }\SpecialCharTok{+} \CommentTok{\# gene expression should actually be on the y axis}
  \FunctionTok{scale\_y\_discrete}\NormalTok{(}\AttributeTok{labels=}\FunctionTok{c}\NormalTok{(}\StringTok{"FALSE"} \OtherTok{=} \StringTok{"Less than 5"}\NormalTok{, }\StringTok{"TRUE"} \OtherTok{=} \StringTok{"More than 5}\SpecialCharTok{\textbackslash{}n}\StringTok{(Extreme)"}\NormalTok{)) }\CommentTok{\# name the x axis by \textgreater{}\textless{}5}
\NormalTok{boxplot}
\end{Highlighting}
\end{Shaded}

\includegraphics{103ProjectNotebook_files/figure-latex/unnamed-chunk-6-1.pdf}

\begin{enumerate}
\def\labelenumi{\arabic{enumi}.}
\setcounter{enumi}{3}
\tightlist
\item
  develop a function that will create these plots with input as: o The
  name of a data frame o A list of 1 or more gene names o 1 continuous
  covariate o A list of 2 categorical covariates
\end{enumerate}

For this function, I decided to plot the data for all the genes of
interest (since it is not specified in the assignment) so there are 3
plots no matter how many genes you put in

\begin{Shaded}
\begin{Highlighting}[]
\NormalTok{make\_plot\_function }\OtherTok{\textless{}{-}} \ControlFlowTok{function}\NormalTok{(df, genes, contvar, list2catvars)\{}
  \CommentTok{\# only select data for relevant genes}
  \ControlFlowTok{for}\NormalTok{ (gene }\ControlFlowTok{in} \FunctionTok{range}\NormalTok{(}\DecValTok{1}\SpecialCharTok{:}\FunctionTok{length}\NormalTok{(genes)))\{ }\CommentTok{\# for each gene, make a df. then add the df to the list}
    \ControlFlowTok{if}\NormalTok{ (gene }\SpecialCharTok{==} \DecValTok{1}\NormalTok{)\{}
\NormalTok{      initialsubsetdf }\OtherTok{=} \FunctionTok{subset}\NormalTok{(df, X }\SpecialCharTok{==}\NormalTok{ genes[gene])}
\NormalTok{      completedf }\OtherTok{=}\NormalTok{ initialsubsetdf}
\NormalTok{      \} }\ControlFlowTok{else}\NormalTok{ \{}
\NormalTok{        newsubsetdf }\OtherTok{=} \FunctionTok{subset}\NormalTok{(df, X }\SpecialCharTok{==}\NormalTok{ genes[gene])}
\NormalTok{        completedf }\OtherTok{=}  \FunctionTok{merge}\NormalTok{(completedf, newsubsetdf, }\AttributeTok{all=}\NormalTok{T)}
\NormalTok{      \}}
\NormalTok{  \}}
  
  \CommentTok{\# get charlson binary since it could be a cat var (ie I did that so I want to add it to my function). It could be simpler of course with the premade cat var}
\NormalTok{  completedf}\SpecialCharTok{$}\NormalTok{charlson\_binary }\OtherTok{\textless{}{-}}\NormalTok{ completedf}\SpecialCharTok{$}\NormalTok{charlson\_score }\SpecialCharTok{\textgreater{}} \DecValTok{5}
  
  \CommentTok{\# make the 3 plots}
  \CommentTok{\# histogram}
  
\NormalTok{  testhist }\OtherTok{\textless{}{-}} \FunctionTok{ggplot}\NormalTok{(completedf, }\FunctionTok{aes}\NormalTok{(}\AttributeTok{x =} \FunctionTok{unlist}\NormalTok{(completedf[contvar]))) }\SpecialCharTok{+} \CommentTok{\# build a histogram plot of gene expression and frequency}
    \FunctionTok{geom\_histogram}\NormalTok{(}\AttributeTok{color =} \StringTok{"pink"}\NormalTok{, }\AttributeTok{fill =} \StringTok{"magenta"}\NormalTok{) }\SpecialCharTok{+}
    \FunctionTok{labs}\NormalTok{(}\AttributeTok{title =} \FunctionTok{paste}\NormalTok{(}\StringTok{"Requested "}\NormalTok{, contvar, }\StringTok{" Frequency"}\NormalTok{), }\AttributeTok{y =} \StringTok{"Frequency"}\NormalTok{, }\AttributeTok{x =} \FunctionTok{paste}\NormalTok{(}\StringTok{"Requested "}\NormalTok{, contvar)) }\SpecialCharTok{+}
    \FunctionTok{theme}\NormalTok{(}\AttributeTok{plot.title =} \FunctionTok{element\_text}\NormalTok{(}\AttributeTok{colour =} \StringTok{"orange1"}\NormalTok{, }\AttributeTok{face =} \StringTok{"bold"}\NormalTok{, }\AttributeTok{hjust =} \FloatTok{0.5}\NormalTok{)) }\CommentTok{\# for the plot title}
  
\NormalTok{  testhist}
  
  \CommentTok{\# scatter plot}
  
\NormalTok{  scatterplot }\OtherTok{\textless{}{-}} \FunctionTok{ggplot}\NormalTok{(completedf, }\FunctionTok{aes}\NormalTok{(}\AttributeTok{x =} \FunctionTok{unlist}\NormalTok{(completedf[contvar]), }\AttributeTok{y =} \FunctionTok{unlist}\NormalTok{(completedf[list2catvars[}\DecValTok{2}\NormalTok{]]))) }\SpecialCharTok{+}  \CommentTok{\# build a scatter plot of contvar and second catvar}
    \FunctionTok{labs}\NormalTok{(}\AttributeTok{title =} \FunctionTok{paste}\NormalTok{(}\StringTok{"Requested "}\NormalTok{, contvar, }\StringTok{" and "}\NormalTok{, list2catvars[}\DecValTok{2}\NormalTok{]), }\AttributeTok{x =}\NormalTok{ contvar, }\AttributeTok{y =}\NormalTok{ list2catvars[}\DecValTok{2}\NormalTok{]) }\SpecialCharTok{+}
    \FunctionTok{geom\_point}\NormalTok{(}\AttributeTok{color =} \StringTok{"purple"}\NormalTok{, }\AttributeTok{alpha =} \FloatTok{0.5}\NormalTok{) }\SpecialCharTok{+}
    \FunctionTok{theme}\NormalTok{(}\AttributeTok{plot.title =} \FunctionTok{element\_text}\NormalTok{(}\AttributeTok{colour =} \StringTok{"darkmagenta"}\NormalTok{, }\AttributeTok{face =} \StringTok{"bold"}\NormalTok{, }\AttributeTok{hjust =} \FloatTok{0.6}\NormalTok{)) }\CommentTok{\# for the plot title}
  
\NormalTok{  scatterplot}
  
  \CommentTok{\# box plot}
  
\NormalTok{  boxplot }\OtherTok{\textless{}{-}} \FunctionTok{ggplot}\NormalTok{(completedf, }\FunctionTok{aes}\NormalTok{(}\AttributeTok{x =} \FunctionTok{unlist}\NormalTok{(completedf[contvar]), }\AttributeTok{y =} \FunctionTok{unlist}\NormalTok{(completedf[list2catvars[}\DecValTok{1}\NormalTok{]]), }\AttributeTok{color =} \FunctionTok{unlist}\NormalTok{(completedf[list2catvars[}\DecValTok{2}\NormalTok{]]))) }\SpecialCharTok{+} \CommentTok{\#create a boxplot of all 3 vars}
    \FunctionTok{geom\_boxplot}\NormalTok{() }\SpecialCharTok{+}
    \FunctionTok{geom\_jitter}\NormalTok{() }\SpecialCharTok{+}
    \FunctionTok{labs}\NormalTok{(}\AttributeTok{title =} \FunctionTok{paste}\NormalTok{(}\StringTok{"Plotting "}\NormalTok{, contvar, list2catvars), }\AttributeTok{x =} \FunctionTok{paste}\NormalTok{(}\StringTok{"Requested gene "}\NormalTok{, contvar), }\AttributeTok{y =} \FunctionTok{paste}\NormalTok{(}\StringTok{"Requested"}\NormalTok{, list2catvars[}\DecValTok{1}\NormalTok{])) }\SpecialCharTok{+} 
    \FunctionTok{theme}\NormalTok{(}\AttributeTok{plot.title =} \FunctionTok{element\_text}\NormalTok{(}\AttributeTok{colour =} \StringTok{"orangered"}\NormalTok{, }\AttributeTok{hjust =} \FloatTok{0.18}\NormalTok{)) }\SpecialCharTok{+} \CommentTok{\# for the plot title}
    \FunctionTok{scale\_fill\_manual}\NormalTok{(}\AttributeTok{labels=}\FunctionTok{c}\NormalTok{(}\StringTok{"\textless{}5"}\NormalTok{, }\StringTok{"\textgreater{}5"}\NormalTok{)) }\SpecialCharTok{+}
    \FunctionTok{coord\_flip}\NormalTok{() }\CommentTok{\# name the x axis by \textgreater{}\textless{}5}
  
\NormalTok{  boxplot}
\NormalTok{\}}
\end{Highlighting}
\end{Shaded}

\begin{Shaded}
\begin{Highlighting}[]
\FunctionTok{make\_plot\_function}\NormalTok{(df, }\FunctionTok{c}\NormalTok{(}\StringTok{\textquotesingle{}CLEC3B\textquotesingle{}}\NormalTok{, }\StringTok{\textquotesingle{}APOA1\textquotesingle{}}\NormalTok{), }\StringTok{"gene\_expression\_data"}\NormalTok{, }\FunctionTok{c}\NormalTok{(}\StringTok{"charlson\_binary"}\NormalTok{, }\StringTok{"mechanical\_ventilation"}\NormalTok{))}
\end{Highlighting}
\end{Shaded}

\includegraphics{103ProjectNotebook_files/figure-latex/unnamed-chunk-8-1.pdf}

Select 2 additional genes to look at and implement a loop to generate
your figures using the function you created

\begin{Shaded}
\begin{Highlighting}[]
\FunctionTok{make\_plot\_function}\NormalTok{(df, }\FunctionTok{c}\NormalTok{(}\StringTok{"APOM"}\NormalTok{, }\StringTok{"CTSG"}\NormalTok{, }\StringTok{"DEFA4"}\NormalTok{), }\StringTok{"gene\_expression\_data"}\NormalTok{, }\FunctionTok{c}\NormalTok{(}\StringTok{"charlson\_binary"}\NormalTok{, }\StringTok{"mechanical\_ventilation"}\NormalTok{))}
\end{Highlighting}
\end{Shaded}

\includegraphics{103ProjectNotebook_files/figure-latex/unnamed-chunk-9-1.pdf}

\end{document}
